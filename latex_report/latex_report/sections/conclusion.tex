\section{Conclusion}

The Apriori rule gives information about the types of degrees combinations of degrees and college features preferred by different ethnic groups. 
Clustering results in comparison and grouping of different features of the dataset and hence result in the comparison between various ethnic groups by using different factors. Unfortunately, the results of clustering are not satisfactory and do not give rise to good results on this type of a dataset. The failure of clustering algorithms can be attributed to the highly dimensional and varied nature (contains highly dimensional categorical features along with a census data) of the dataset. 
The classification task gives rise to a logistic regression model which can predict the presence of a person from a certain ethnic group in a college degree based on the features of the degree and also the existing ethnic groups in the college degree. The neural network approach did not yield good results but can be a promising project in the near future.
To conclude, the three data mining techniques applied give us different valuable insights into the patterns in higher education enrollment in the country. These results can be used for a better analysis and understanding of the education trends of different ethnic groups and genders in India. Moreover, this paper deals with the analysis of and implementation of 3 data mining techniques on a large, sparse, high dimensional, categorical, and census data, and hence serves as a guide for applying basic data mining tasks on such datasets.