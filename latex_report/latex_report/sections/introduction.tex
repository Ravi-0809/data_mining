\section{Introduction}
This research project aims to analyze the student enrollment patterns in higher education programmes across India in the year of 2015. The aim is to analyze and find relationships and patterns between different factors that affect enrollment in a particular degree or college. The dataset is obtained from Indian government data website \cite{b1}. This dataset contains 400,000+ entries of programmes which show the types of students enrolled in different higher education programmes. 

Different pre-processing techniques are applied on the dataset to make the data more suitable for use in visualizations as well as data mining tasks. After visualizing the data, data mining tasks are performed on the dataset to obtain the desired relationships and patterns. Three data mining techniques are explored in this paper and are applied on this data set. The three techniques are as follows:
\begin{itemize}
    \item Association rule mining using the Apriori algorithm \cite{b2}
    \item Clustering using the K-means algorithm \cite{b3}
    \item Classification using logistic regression and a neural network
\end{itemize} 

All the pre-processing techniques, visualizations, and all the three data mining techniques were implemented in python3 with the help of jupyter notebooks and the complete implementation of this paper can be found on GitHub.