\section{Problem Definition}
The aim of this project is to find patterns and draw insights from student enrollment statistics in higher education programmes in India for the year 2015. We aim to use this dataset to draw the following types of insights mentioned below. The groups of people refers to whether a person is from the general caste, backward castes, PWD, muslim minority, or other minorities.
\begin{itemize}
    \item The relation between different groups of people and the type of higher education they prefer
    \item The relation between different groups of people and the location in which they take up their education
    \item The relation between different groups of people and the number of years they wish to study
    \item The relation between females and the type of education and hostels they prefer
    \item The relation between one group of people and another in terms of taking up enrollment in that degree
\end{itemize}

Based on the patterns and relations we wish to draw, three data mining techniques are used. The three techniques are association rule mining using Apriori rule, clustering using K-means and classification using logistic regression and a neural network approach. 