\section{Data description}
The chosen dataset contains details about different higher education programmes in India offered in the year of 2015. The dataset contains 400,000+ entries  of different colleges and programmes offered by these colleges. The dataset contains information about 35,000 different colleges in India and 178 different programmes offered by them. Programmes refer to different types of degrees like B.E Computer Science, B.Sc. Physics, and a variety of undergraduate, post-graduate and phd degrees. The degrees are from multiple disciplines like art, commerce, science, political science and many more. There are 19651 disciplines present in the dataset. 

The dataset initially (before pre-processing) contained 58 features. These features contain information about the demographic of students in a particular college in a particular degree in the year 2015. The demographic of the students gives us information like what caste they are from (General, SC, ST, OBC), what minority they are a part of if any (PWD, muslim, other minorities), and also the number of females and males in each of these minorities and castes. The dataset contains numerical values on the enrollment of members from each of these castes, minorities, gender, and also gives us numerical values of the combinations of the above three mentioned splits in the demographic. For example, a mixed feature would represent the number of people enrolled in a particular college, in a particular degree who are muslim, SC as well as female. 
Apart from the demographic, the dataset contains more information about each programme like which broader discipline they belong to, the minimum number of years needed to complete the degree, if the degree is self financed or not, etc. 

To add more information about colleges to this dataset, another dataset \cite{b5} was chosen from the Indian government data website and merged with the initial dataset. This additional dataset contained valuable information about the geographical location of the college (the state and the city in which the college is located), the specialty of the degree wherever applicable, and also if the college provides hostel facilities or not. With this, 4 additional features were added to the original dataset giving a total of 62 features before pre-processing.  

This dataset can also be understood by splitting into two parts - A part containing the demographic of students in the college (census of each degree in each college), and a part containing different features of each college and degree like discipline, location, etc. The first part is census data or ratio data. The second part is a highly dimensional categorical data. Another important point to note about this dataset is that the data is very sparse and contains a lot of missing values and zeros. These problems are tackled accordingly while pre-processing the dataset. 